\begin{abstract}
Approximate similarity search in graph databases retrieves graphs whose \emph{graph edit distance} ($\ged$) to a given query graph is within a given threshold. As the $\ged$ computation is NP-hard, indexing has been proposed to accelerate the query by reducing the number of such computations. As a fundamental property for indexing metric spaces, the concept of \emph{doubling metrics} has been proposed for similarity search in some metric spaces, but it has not been studied in similarity search in graph databases.
In this work, we are the first to study doubling metrics in the $\ged$ space. We show that the $\ged$ space is \emph{doubling} — any ball of radius $r$ can be covered by a bounded number (the doubling constant) of balls of radius $r/2$. Importantly, we reveal its two-stage phenomenon. First, when the $\ged$ exceeds a threshold $\alpha$, the doubling constant is small. Doubling-based indexes are efficient, and the search approaches candidate query answers efficiently in the $\ged$ space. Second, when the $\ged$ is below $\alpha$, the doubling constant grows rapidly, making doubling-based indexes inefficient and requiring many $\ged$ computations in the subspace near the answers.
Based on these insights, we propose \textbf{G}iant-\textbf{S}tep-\textbf{S}mall-\textbf{S}tep ($\net$), a novel two-stage index for approximate graph similarity search. The first stage of $\net$, namely $\upperpart$, is a doubling-based index to facilitate a {\em giant-step search} having the lowest known time complexity, while the second stage of $\net$, called $\lowerpartfull$ ($\lowerpartshort$), performs a traversal in {\em small (search) steps} to achieve both high efficiency and recall. Experiments on four benchmark datasets demonstrate that $\net$ outperforms state-of-the-art solutions in both efficiency and effectiveness.
\end{abstract}
