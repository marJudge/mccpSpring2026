\section{Preliminaries and Problem Statement}
\label{sec:preliminaries}
In this section, we introduce the preliminaries of approximate similarity search in graph databases. Then, we present some background and an overview of 
%our solution 
our indexing framework $\net$. 
For clarity, we summarize the frequently used notations in Table~\ref{symbols}.




\subsection{Preliminaries}
\label{subsec:preliminaries}
We study a database of undirected graphs with vertex labels but without edge labels, following prior work~\cite{LAN,ghash,cheng2025computing,GEDGNN}. This setting is compatible with existing graph similarity search methods that incorporate edge labels~\cite{LBMaTKDE22,NassGED,xu2025graph}.

\stab
\noindent
\textbf{Graph.}
A graph is denoted by $G = (V, E, \ell)$, where $V$ and $E$ are the sets of vertices and undirected edges of $G$, respectively.
The mapping $\ell: V \to L$ is a function that maps a vertex in $V$ to a label.
For clarity, we use $V(G)$ and $E(G)$ to denote the vertex set and edge set of a graph $G$, respectively.


\stab
\noindent
\textbf{Graph edit path and distance~\cite{ghash, noah, LAN, GEDGNN, cheng2025computing}}.
We consider five edit operations on graphs, including vertex insertion, vertex deletion, vertex relabeling, edge insertion, and edge deletion. Given two graphs $G$ and $G'$, a \emph{graph edit path} (GEP) from $G$ to $G'$ is the \emph{shortest} sequence of edit operations on $G$ that transforms $G$ into $G'$; and the \emph{graph edit distance} (GED) between $G$ and $G'$, denoted by $d(G,G')$, is the length of the GEP from $G$ to $G'$.


\begin{example}
\noindent
Figure~\ref{fig:combined_rnet_editpath}(b) shows a graph edit path from $G$ to $G'$ with the sequence $[O_1,O_2,O_3]$ of operations, where 
(i) $O_1$ is the vertex insertion of vertex $v_5$ with label $S$; 
(ii) $O_2$ is the edge insertion of edge $(v_1, v_5)$; 
(iii) $O_3$ is vertex relabeling that changes $v_4$'s label from $C$ to $N$. 
As $[O_1,O_2,O_3]$ has the minimum length of $3$, the graph edit distance between $G$ and $G'$ is $d(G,G')$ $=$ $3$.
\end{example}



% \lyu{According to~\cite{serratosa2019graph}, GED can be considered as a metric.}\lv{Yikai:what is the purpose of this sentence?}

\noindent
Computing the exact $\ged$ is NP-hard~\cite{starVLDB09}, which makes graph similarity search computationally costly on large databases. 
Therefore, in this paper, we propose an 
index-based \emph{approximate graph similarity search} solution
%\lyu{indexing framework for \emph{approximate graph similarity search}}
to efficiently tackle this problem.


\stab
\noindent
\textbf{Problem statement.} 
Given a graph database $\mathcal{D} = \{G_1, \ldots, G_n\}$, a query graph $Q$ and a threshold $\tau$, 
an \emph{approximate graph similarity search} problem is to retrieve graphs $G$ in $\mathcal{D}$ such that $d(G, Q)$ ($\ged$ between $G$ and $Q$) is at most $\tau$. This paper aims to provide an efficient solution with a high recall.



\subsection{Background of $\net$}

To present $\net$, we list several fundamental concepts in metric spaces, including 
the doubling constant~\cite{navnet}, the expansion constant~\cite{KR2002,CoverTree}, 
doubling spaces~\cite{navnet}, and $r$-nets~\cite{navnet,NetTree}.
We first recall the definition of a ball  
in a metric space. A ball $B(p,r)$ is the set of all points whose distance to the center $p$ does not exceed the radius $r$, and we use $|B(p,r)|$ to denote the number of points covered by $B(p,r)$.




\begin{definition}[\emph{Doubling constant}~\cite{navnet}]
\label{def:doubling_constant}
Given a metric space $(\mathcal X,d)$, the \emph{doubling constant} $\lambda$ is 
the smallest number such that every ball $B(p,2r)$ can be covered by at most 
$\lambda$ balls of radius $r$, for any $p\in\mathcal X$ and $r>0$:
\[
\lambda = \max_{p\in\mathcal X,\, r>0} \lambda(p,r),\;
\text{where }\lambda(p,r)=\min\{k: B(p,2r)\subseteq \cup_{i=1}^k B(p_i,r)\}.
\]
For a given threshold of a radius $\alpha>0$, $\lambda$ of $\alpha$ is defined as:
\[
\lambda_{\alpha}:= \max_{p\in\mathcal X,\, r\ge \alpha} \lambda(p,r).
\]
\end{definition}




\begin{definition}[\emph{Expansion constant}~\cite{KR2002,CoverTree}]
\label{def:expansion_constant}
Given a metric space $(\mathcal{X},d)$, the \emph{expansion constant} $c$ is the smallest value $c \ge 2$ such that
\[
  |B(p,2r)| \le c \cdot |B(p,r)| \quad \text{for all } p \in \mathcal{X},\; r>0.
\]
It can also be stated below:
\[
  c = \max_{p\in\mathcal X,\, r>0} c(p,r),
  \quad \text{where } c(p,r)=\frac{|B(p,2r)|}{|B(p,r)|}.
\]
For a given threshold of a radius $\alpha>0$, $c$ of $\alpha$ is defined as:
\[
c_{\alpha} := \max_{p\in\mathcal X,\, r\ge \alpha} c(p,r).
\]
\end{definition}



\begin{definition}[\emph{Doubling space}~\cite{navnet, NetTree}]
\label{def:doubling_space}
A metric space $(\mathcal{X}, d)$ is \emph{doubling} if its doubling constant $\lambda$ is finite.
\end{definition}

\begin{example}[Computing $\lambda(p,r)$ and $c(p,r)$]
\label{ex:doubling}
In Figure~\ref{fig:r-net}, for $p_4$ and radius $r>0$, $B(p_4,2r)$ can be covered by six radius-$r$ balls, so $\lambda(p_4,r)=6$.
Moreover, $|B(p_4,r)|=4$ and $|B(p_4,2r)|=15$. Hence, $c(p_4,r)=15/4=3.75$.
Taking the maximum over all $p$ and $r$, we yield the doubling constant $\lambda$ and expansion constant $c$.
\end{example}




\noindent
\begin{definition}
\label{def:r-net}
(\emph{$r$-net}~\cite{navnet})
Given a metric space $(\mathcal{X}, d)$, where $\mathcal X$ is a set of points and $d$ is the distance measure, an \emph{$r$-net} $Y \subseteq \mathcal{X}$ is a subset of $\mathcal{X}$ that satisfies the following:
\begin{itemize}[wide, labelwidth=!, labelindent=0pt]
  \item For $p_i,p_j \in Y$, $p_i \neq p_j$ $\Rightarrow$ $d(p_i,p_j) \ge r$; and
  \item $\mathcal{X} = \bigcup_{p \in Y} B(p,r)$, where $B(p,r)=\{p'\in \mathcal{X} \mid d(p, p') \le r\}$.
\end{itemize}

$r$ denotes the radius of an $r$-net. 
Intuitively, the first condition ensures that the points in $Y$ are \emph{$r$-separated}~\cite{NetTree}, 
meaning that their pairwise distances are at least $r$, 
and the second condition ensures that the collection of $r$-balls $\{B(p,r)\mid p\in Y\}$ covers the entire space $\mathcal{X}$.
\end{definition}



\begin{example}
Figure~\ref{fig:r-net} shows the space $\mathcal{X}$, where each node is a data point in $\mathcal{X}$.
$Y=\{p_1, p_2, \ldots, p_6\}$ forms an $r$-net of $\mathcal{X}$.
The balls of radius $r$ centered at $p_i$ for $i \in \{1,\ldots,6\}$ together cover all points, and the centers satisfy $d(p_i,p_j) > r$ for any $i \ne j$.
\end{example}

Net-tree~\cite{NetTree} presents an iterative procedure to construct a $r$-net. We briefly summarize its relevant background here.

\noindent
\begin{definition}
\label{def:greedy-permutation}
(\emph{Greedy permutation}~\cite{NetTree})
Given a metric space $(\mathcal{X}, d)$, an ordering of centers
$\langle p_1, p_2, \ldots, p_N \rangle$,
where $N = |\mathcal{X}|$, is a \emph{greedy permutation} of 
$\mathcal{X}$ if for every $k \ge 2$,
\[
    p_k = \arg\max_{p\in\mathcal X} \min_{1\le i<k} d(p, p_i).
\]
%Each $p_k$ is referred to as a \emph{center}.
\end{definition}


%The sequence $r_1 \ge r_2 \ge \cdots \ge r_N$ is non-increasing.  



Net-tree~\cite{NetTree} generates a greedy permutation by generating the centers iteratively together with their distances $\{r_k\}$ to the nearest previously chosen center. For each $k$, 
the distance from $p_k$ to its nearest earlier center, $r_k$, is defined as $\min\limits_{1 \le i < k} d(p_k, p_i)$. Also, each center $p_i$ maintains a {\em friends list} that 
contains all centers whose distance to $p_i$ is at most 
$\min\{8 r_k, 4 r_i\}$ at the $k$th iteration, which will be used to compute the permutation.
Importantly, the prefix $\langle p_1,\ldots,p_k\rangle$, when casted into a set 
$\{p_1,\ldots,p_k\}$, forms a $r_k$-net of $\mathcal{X}$ 
(Definition~\ref{def:r-net}).
An $r$-net is obtained at the end of the iterative procedure.




\begin{table}[t]
\renewcommand{\arraystretch}{1.2}
\footnotesize  
    \setlength{\tabcolsep}{3pt}  
    \vspace{-3.5ex}
\caption{Frequently used notations}
\centering
\begin{tabularx}{\columnwidth}{|c|X|}
\hline
\textbf{Notation} & \textbf{Description}\\ 
\hline\hline
$\mathcal{D}$ & A database containing multiple small (data) graphs\\
\hline
$Q$; $G$ & A query graph $Q$; a (data) graph $G$ in $\mathcal{D}$\\
\hline
$(Q,\tau)$ & A similarity search query with range $\tau$\\
\hline
$V(G)$; $E(G)$ & The vertex set and edge set of a graph $G$, respectively\\
\hline
$(\mathcal X, d)$ & A metric space where $\mathcal X$ is a set of points (graphs) and $d$ is the distance measure \emph{(in this paper, $d$ is the graph edit distance)}\\
\hline
$d(G, G')$ & The distance $d(\cdot,\cdot)$ evaluated between two graphs $G, G'$, which is \emph{the graph edit distance ($\ged$) in our scenario}\\
\hline
$B(p,r)$ & The (closed) ball centered at $p$ with radius $r$, 
i.e., the set of all points within distance $r$ from $p$: 
$\{\, p' \in \mathcal{X} \mid d(p,p') \le r \}$\\
\hline
$|B(\cdot,\cdot)|$ & The number of points (graphs) within a ball $B(\cdot,\cdot)$\\
\hline
$r$-ball & Any ball $B(p,r)$ of radius $r$, where the specific center $p$ is not specified\\
\hline
$\ell(v)$ & The label of vertex $v$ in a data graph $G$\\
\hline
$N$ & The number of data graphs in the database\\
\hline
$\tau$ & The threshold parameter for similarity queries\\
\hline
$\epsilon$ & The error tolerance of the approximate GED method\\
\hline

$\lambda(p,r)$ & The minimum number of balls of radius $r$ to cover $B(p,2r)$ for a given $p$ and $r$\\
\hline
$c(p,r)$ & The ratio $|B(p,2r)|/|B(p,r)|$ for a given $p$ and $r$ \\
\hline

$\lambda$ & A doubling constant (ball-cover-based) \\
\hline
$c$ & An expansion constant (node-count-based) \\
\hline
$p_j^i$ & The representative of database point $p_j$ in layer $Y_i$ of $\upperpart$\\
\hline
\end{tabularx}
\label{symbols}
\end{table}





\begin{figure*}[tb]
    \captionsetup{width=1\linewidth}
    \centering
    
    \begin{subfigure}[t]{0.43\textwidth}
        \centering
        \includegraphics[width=\linewidth]{figures/index_introduction/overview.png}
        \caption{Overview of $\net$}
        \label{fig:overview}
    \end{subfigure}
    \hfill
    \begin{subfigure}[t]{0.29\textwidth}
        \centering
        \includegraphics[width=\linewidth]{figures/index_introduction/ND_overview.png}
        \caption{An example of $\upperpart$}
        \label{fig:ND_overview}
    \end{subfigure}
    \hfill
    \begin{subfigure}[t]{0.26\textwidth}
        \centering
        % \includegraphics[width=\linewidth]{figures/index_introduction/EPT_overview.png}
        \includegraphics[width=0.8\linewidth]{figures/index_introduction/subtree-prune-example.png}
        \caption{An example of $\lowerpartsinglefull$}
        \label{fig:EPT_overview}
    \end{subfigure}
    \vspace{-2ex}
    \caption[Overview of $\net$]{Overview of $\net$ -- Stage 1: $\upperpart$ (Section~\ref{sec:ND}); and Stage 2: $\lowerpartfull$ (Section~\ref{sec:EPT}). Both indices depend on the threshold $\alpha$, which serves both as the radius of the $r$-net at $\upperpart$’s leaf layer and as the radius of the balls indexed by $\lowerpartsingleshort$s. $\upperpart$ operates at larger $\ged$s, while $\lowerpartfull$ handles smaller $\ged$s.}
    \label{fig:above_search_algo}
    \vspace{-5ex}
\end{figure*}


\subsection{$\net$ Overview}

Figure~\ref{fig:overview} provides an overview of our proposed two-stage indexing framework $\net$.
In particular, 
the first stage of $\net$, namely $\upperpart$, consists of multiple layers of $r$-nets, where the radius of the $r$-net in a parent layer is twice that of its child layer. Between consecutive layers, an {\em index node (a ball center)} in the parent layer may have edges to nodes in the child layer that fall within a certain range (see Section~\ref{sec:ND} for details), forming a directed acyclic graph (DAG).
The leaf layer of $\upperpart$ is an $\alpha$-net $Y_{\alpha}$.
The union of balls centered at the points in $Y_{\alpha}$ with radius $\alpha$ covers the entire graph database. In the second stage, an $\lowerpartfull$ consists of a set of $\lowerpartsinglefull$s ($\lowerpartsingleshort$s). An $\lowerpartsingleshort$ is a tree rooted at $p$, where $p$ is in $Y_{\alpha}$. $\lowerpartsingleshort$ indexes the graph edit paths between the graphs in $B(p, \alpha)$ and $p$.
%We remark that a graph may be covered by multiple balls, and it is simply assigned to the one whose center is closest.

During query processing, we traverse from the top layer of $\upperpart$ to its leaf layer. 
In each layer, we determine a node whose children satisfy a specific condition such that their balls collectively cover $B(Q,\tau)$ (see Section~\ref{sec:ND_query} for details). This enables the traversal to proceed to one of the children in the lower layer. 
When the traversal reaches the leaf layer, we directly determine which balls jointly cover all answers. 
Next, we identify the $\lowerpartsingleshort$s of the balls, which cover the answer candidates, and perform a depth-first search on each of them. 
To further improve efficiency, three optimizations are proposed for the search for $\lowerpartshort$.

It should be noted that the parameter $\alpha$ indicates the boundary between $\upperpart$ and $\lowerpartfull$. 
In Section~\ref{subsec:alpha-tradeoff}, we propose a cost model to determine an optimal $\alpha$ for a graph database.
