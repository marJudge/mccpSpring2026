\section{Introduction}
% Graphs have been widely used in a lot of emerging applications, including bioinformatics, chemi-informatics, software engineering, and computer vision. Nearest neighbor (NN) search is a fundamental query in graph databases \cite{}. Graph edit distance (GED) is one of the most widely used distance measures in NN search \cite{bai2019simgnn, noah, ghash, LAN, inves, nass, LSaICDE20, LBMaTKDE22}. However, exact NN search in a large database is too time consuming due to the NP-hardness of GED computation. Therefore, approximate NN (ANN) search becomes a promising alternative as it can significantly reduce the time cost with a slight reduce of accuracy.  

Similarity search in graph databases is to search for data graphs that are similar to a given query graph. It is a fundamental query with
%in graph databases and has 
a wide range of applications, including bioinformatics, chemi-informatics, software engineering, and computer vision \cite{MCP, ChemExample, parsk, LigandGED, naderi2016graph, zhuTKDE20, GEDdynamicEmb, DFGED, ghash, SoftworeExample, SEGOS}. 
For example, %researchers
biologists might want to find the molecules in a compound database having a similar molecular structure as that of
%similar to
a given drug molecule, where the structures of compounds can be considered as graphs. 
The
\emph{graph edit distance} (GED) has been a widely used 
similarity metric
%distance measures 
in graph similarity search \cite{bai2019simgnn, noah, ghash, LAN, inves, NassGED, LSaICDE20, LBMaTKDE22}. However, due to the NP-hardness of the computation of (exact) GED, similarity search on a large graph database is computationally expensive. Hence, approximate similarity search has been proposed for practical graph similarity search. In this paper, we propose a new indexing scheme for approximate similarity search in graph databases.
 %focus on 
%\lyu{consider} graph databases \lyu{consisted} of small \lyu{and} medium-sized graphs \cite{aids, pubchem}.




\begin{figure}[t]
%\captionsetup{width=1.1\linewidth}
\begin{subfigure}{4cm}
% \vspace{0.5ex}
%\hspace{-3ex}
		\includegraphics[width=4.2cm]{figures/ged_dist/ged_dict_15node_lineplot_40.png}\hfill
		\vspace{-3.5ex}
		\caption{Distribution of GEDs between graphs in AIDS having a number of vertices not larger than~$15$}
		\vspace{-2ex}
\end{subfigure}
\hspace{1ex}
\begin{subfigure}{3.8cm}
\vspace{0ex}
%\hspace{-3ex}
		\includegraphics[width=4.2cm]{figures/ged_dist/all_ged_lineplot_200.png}\hfill
		\vspace{-3.5ex}
		\caption{Distribution of approximate GEDs between graphs in AIDs by using GREED~\cite{GREED}}
		\vspace{-2ex}
\end{subfigure}
\caption{Distributions of the graph edit distances (GEDs) between graphs in the AIDS dataset under different settings}

\label{histo}
\vspace{-2ex}
\end{figure}


%\lv{start here}

There have been works on applying similarity search/approximate $k$-nearest neighbor ($\aknn$) search in metric spaces ${\mathcal U}({\mathcal G}, d)$ to graph similarity search, as GED is a metric \cite{serratosa2019graph}. 
Existing methods can be described with two extremes of the assumptions on the {\em expansion rate}, which are illustrated with the horizontal dimension of Figure~\ref{fig:dimsetting}, \choi{where the expansion rate $c$ of XXX is defined as ...} of the spaces. 
\choi{At one extreme, the expansion rate can be bounded by a constant.} A doubling space \cite{KL2004navigating} states that a ball $B(p, 2r)$ can be covered by $C_{KL}$ balls with radius $r$, where the expansion rate $C_{KL}$ is a {\em constant} and a ball $B(p, r)$ denotes all the points $p'$ such that $d(p, p') \leq r$ 
(\eg~all the points $p'$ in the ball of center $p$ and radius $r$). By exploiting the fact that the expansion rate is constant, the AKNN method uses a hierarchical structure in this space. $\tt NavNet$ \cite{navnet} and $\tt CoverTree$ \cite{CoverTree} and their variants \cite{compressedCoverTree, gu2022parallel, elkin2023new} are the latest methods that requires this assumption. Their search time complexity is $O$($lg$($n$)), where $n$ is the number of objects in the database. However, the expansion rate may not be bounded by a constant in practice. 

At the other extreme, the methods assume that the expansion rate is not bounded. These do not utilize the \choi{expansion properties} of the space. In particular, $\tt HNSW$ \cite{hnsw}, $\tt MRNG$ \cite{nsg}, and $\tau$-$\tt MG$ \cite{tauMG} are the latest works of AKNN search. However, the search accuracy and search time complexity of $\tt HNSW$ has no guarantee. Although $\tt MRNG$ and $\tau$-$\tt MG$ find the exact NN $R$ of $Q$ when $d(Q,R)=0$ and $d(Q,R)<\tau$, respectively, the NDCs of $\tt MRNG$ and $\tau$-$\tt MG$ are $O(N)$ time in metric spaces. 

There has been another assumption $(\rho, C_{KR})$-expansion \cite{KR2002}. Specifically, a space ${\mathcal U}$ has $(\rho, C_{KR})$-expansion \cite{KR2002} means that $|B(p, r)|\geq \rho$ $\implies$ $|B(p, 2r)|\leq C_{KR}|B(p, r)|$ for any point $p \in {\mathcal U}$, $r>0$ and $C_{KR}$ is a constant, where a ball $B(p, r)$ denotes all the points $p'$ such that $d(p, p') \leq r$ (e.g. all the points $p'$ in the ball of center $p$ and radius $r$), and $|B(\cdot)|$ means the number of points in the ball. 

%\lv{from [start here] to here: Can you conclude high-level ideas why we study the following practical setting? Actually, there should be a flow for paragraphs 2-8 in Section 1 to introduce why and how we study the problem.}


In this paper, we study the approximate graph similarity search with the GED metric in a practical setting between the two extremes: {\it the expansion rate $c$ is a monotonically increasing function of $n$}, where $n$ is the size of the largest graph in $\mathcal G$. It is motivated by our observation of real-world datasets. As shown in Figure~\ref{histo}(a), GED follows an unimodal distribution. Let $\alpha$ be the GED value at the peak. We can observe that when $d>\alpha$, $c=0.5$; when $d<\alpha$, $c=O(n^{2\alpha})$. From Figure~\ref{histo}(b), we can observe that $\alpha$ is stable with the growth of $n$, {\it i.e.}, $\alpha$ can be regarded as a constant. 


\eat{
\begin{figure}[t]
\centering
\includegraphics[scale=0.35]{figures/dimsetting}
\caption{A comparison of \yk{$c=O(\alpha log_2n)?$} ($N$ is the number of objects in the database.)}
\vspace{-1ex}
\label{fig:dimsetting}
\vspace{-2ex}
\end{figure}
}




% Existing PGs supporting ANN search in metric spaces assume that only the triangle inequality property is available. $\tt HNSW$ \cite{hnsw}, $\tt MRNG$ \cite{nsg} and $\tau$-$\tt MG$ \cite{tauMG} are the latest works. Let $N$ denote the database size. The number of distance computations (NDC) of the ANN search using $\tt HNSW$ is empirically polylogarithm of $N$. However, the search accuracy of $\tt HNSW$ has no guarantee. $\tt MRNG$ and $\tau$-$\tt MG$ assure to find the exact NN $R$ of $Q$ when $d(Q,R)=0$ and $d(Q,R)<\tau$, respectively. However, the NDCs of $\tt MRNG$ and $\tau$-$\tt MG$ are $O(N)$. 


% Existing PGs fall into two extremes. At one extreme, the PGs are designed for the ANN search in the general metric space, where $\tt HNSW$ \cite{hnsw},  is the latest work. Although $\tt HNSW$  empirically has polylogarithm search time of the database size $N$, the search accuracy of $\tt HNSW$ has no guarantee. At the other extreme, the PGs are designed for the ANN search in the Euclidean space, where $\tt MRNG$ \cite{nsg} and $\tau$-$\tt MG$ \cite{tauMG} are the latest works. Although $\tt MRNG$ and $\tau$-$\tt MG$ can be adapted to the general metric space while assuring to find the exact NN of $Q$ if $d(Q,R)=0$ and $d(Q,R)<\tau$, respectively, the search time complexities of $\tt MRNG$ and $\tau$-$\tt MG$ are $O(N)$. 


% In this paper, we study a practical setting, where the metric space has a bounded doubing dimension. It is motivated by our observation on real-world graphs that the distribution of the upper bound of GED first increases and then decreases with the growth of the largest graph size $n$, and the value $\alpha$ of the upper bound of GED for achieving the peak of the distribution is stable after $n$ is large enough. Figure~\ref{} illustrates the phenomenon for the AIDS dataset. Considering the number of graphs within GED 1 from a graph is $O(n^2)$, it is derivable that  the doubling dimension of the metric space formed by $\mathcal D$ and GED is $O(\log_2 n^{2\alpha})$. $\tt CoverTree$ \cite{CoverTree} is the latest method for ANN search in the metric space with bounded doubling dimension and assures to find the exact NN. However, $\tt CoverTree$ suffers from a high time complexity $O(n^{24\alpha}\log N)$ in graph databases.





% Let $X$ and $Y$ be the random variables of graph size and GED of any two graphs, respectively. $Y$ quadratically depends on $X$, as the GED of two graph is the square of the size of the larger graph in the worst case. $Pr(Y=x^2)\leq Pr(X=x)Pr(X=x) + Pr(X=x)$. If $X$ is normal distributed, $Y$ has the same trend with $X$.


% In this paper, we propose two novel PGs to address the limitation of $\tt CoverTree$. First, we propose a basic method \basic{}. The main idea of \basic{} is that 


% Then, we propose an advanced method \adv{}. The main idea of \adv{} is that 

While a related work $\CoverTree$ and this paper are both positioned in between the two extremes mentioned earlier,
we start by analyzing the reason for the inefficiency of $\CoverTree$. It is due to the {\it geometric} logic of $\CoverTree$, which searches for balls of radius $r$ in balls of radius $2r$. In the GED space, when $r$ is relatively large, $c$ remains a fairly small constant. However, as $r$ decreases to a smaller value, $c$’s value increases sharply, which in turn increases the branching factor in $\CoverTree$ search. This reduces the search efficiency of  $\CoverTree$ and thus suggests that as $r$ is small for the GED space, there is a need to revisit the design of the index.





In this paper, we propose a new index, called $\net$. $\net$ comprises two parts. When $d \geq \alpha$, we utilize the doubling property of the GED space to build a hierarchical index $\upperpart$. In this index, each layer is an $r$-net (mentioned in Definition \ref{def:rnet}). The value of $r$ from top to bottom is $2^{l-1}\alpha, 2^{l-2}\alpha, \ldots, 2\alpha, \alpha$, totaling $l$ layers. Every point $p$ in this $r$-net is the center of a ball $B(p, r + 2\tau_q)$, where $\tau_q$ is the similarity search range. During the search, our method theoretically ensures that all results lie within $B(p, r + 2\tau_q)$. As $r$ decreases along the layers, the final candidate range narrows accordingly. For $d < \alpha$, we build an index called $\lowerpartfull$ ($\lowerpartshort$) for each point in the $\alpha$-net (i.e., the last layer of $\upperpart$). An $\lowerpartshort$ is a tree formed by the point in the $\alpha$-net as the root and its nearby points. Based on the range determined by the search in $\upperpart$, we identify which $\lowerpartshort$ indices may contain the answer, and then perform a fine-grained search with several pruning operations in these $\lowerpartshort$.





% The main idea is that we construct a hierarchy of $r$-nets for $r=1,2,4,8,...,n$. The nodes in the $r_i$-net have directed edges to the nodes in the $r_{i-1}$-net.  
% $\tt NetDag$ assures to find the exact NN in $O(n^{12\alpha}\log N + n^{6\tau})$ time when $d(Q, R)<\tau$. Morevoer, when $d(Q,R)<\infty$, $\tt NetDag$ can find the 2-opt result in  $O(n^{12\alpha}\log N)$ time. 





% \begin{table*}
% \centering
% \begin{tabular}{|l|c|c|c|c|c|c|}\hline
% Method  & $Q$'s setting & index time & index size & query time & guarantee \\\hline
% \hline
% $\tt HNSW$ & any $Q$ & $O(N^2)$ & $O(N\log N)$ & $O(N)$ & None\\\hline
% $\tt MRNG$ & $d(Q, R)=0$ & $O(N^2)$ & $O(N^2)$ & $O(N)$ & Exact\\\hline
% $\tau$-$\tt MG$ & $d(Q, R)<\tau$ & $O(N^2)$ & $O(N^2)$ & $O(N)$ & Exact\\\hline
% $\tt CoverTree$ & any $Q$ & $O(c^6 N\log N)$ & $O(N)$ & $O(c^{12} \log N)$ & Exact\\\hline\hline
% $\tt NetDag$ & $d(Q, R)<\tau$ & $O(c^6 N\log N)$ & $O(c^6 N \log N)$ &  $O(c^{6} \log N + n^{6\tau})$ & Exact\\\hline
% $\tt NetDag$ & any $Q$ & $O(c^6 N\log N)$ & $O(c^6 N \log N)$ &  $O(c^{6} \log N)$ & 2-opt\\\hline
% \end{tabular}
% \caption{Summary on related work ($c$ denotes the doubling constant of the space, $N$ denotes the size of the database, $n$ denotes the size of the largest graph in the database)}
% \label{tab:sumRW}
% \end{table*}



% \begin{table*}
% \centering
% \begin{tabular}{|l|c|c|c|c|c|c|}\hline
% Method  & $Q$'s setting & Index time & Index size & Query time & Guarantee \\\hline
% \hline
% $\tt HNSW$ \cite{hnsw} & any $Q$ & $O(N polylog(N))^*$ & $O(N polylog(N))^*$ & $O(polylog(N))^*$ & None\\\hline
% $\tt MRNG$ \cite{nsg} & $d(Q, R)=0$ & $O(N^2)$ & $O(N^2)$ & $O(N)$ & Exact\\\hline
% $\tau$-$\tt MG$ \cite{tauMG} & $d(Q, R)<\tau$ & $O(N^2)$ & $O(N^2)$ & $O(N)$ & Exact\\\hline
% $\tt CoverTree$ \cite{CoverTree} & any $Q$ & $O(c^{6} N\log N)$ & $O(N)$ & $O(c^{12} \log N)$ & Exact\\\hline\hline
% $\tt NetDag$ & $d(Q, R)<\tau$ & $O(c^{6} N\log N)$ & $O(c^{6} N \log N)$ &  $O(c^{6} \log N + n^{6\tau})$ & Exact\\\hline
% $\tt NetDag$ & any $Q$ & $O(c^{6} N\log N)$ & $O(c^{6} N \log N)$ &  $O(c^{6} \log N)$ & 2-opt\\\hline
% \end{tabular}
% \caption{Summary on related work ($c$ denotes the doubling constant of the metric space, $N$ denotes the size of the database $\mathcal D$, $n$ denotes the size of the largest graph in $\mathcal D$, $\alpha$ and $\tau$ are constants, $R$ is the exact NN of $Q$ in $\mathcal D$, $^*$ means empirical analysis without rigorous proof)}
% \label{tab:sumRW}
% \end{table*}


% \begin{table*}
% \centering
% \begin{tabular}{|l|c|c|c|c|c|c|}\hline
% Method  & $Q$'s setting & Index time & Index size & Query time & Guarantee \\\hline
% \hline
% $\tt HNSW$ \cite{hnsw} & any $Q$ & $O(N polylog(N))^*$ & $O(N polylog(N))^*$ & $O(polylog(N))^*$ & None\\\hline
% $\tt MRNG$ \cite{nsg} & $d(Q, R)=0$ & $O(N^2)$ & $O(N^2)$ & $O(N)$ & Exact\\\hline
% $\tau$-$\tt MG$ \cite{tauMG} & $d(Q, R)<\tau$ & $O(N^2)$ & $O(N^2)$ & $O(N)$ & Exact\\\hline
% $\tt CoverTree$ \cite{CoverTree} & any $Q$ & $O(n^{12\alpha} N\log N)$ & $O(N)$ & $O(n^{24\alpha} \log N)$ & Exact\\\hline\hline
% $\tt NetDag$ & $d(Q, R)<\tau$ & $O(n^{12\alpha} N\log N)$ & $O(n^{12\alpha} N \log N)$ &  $O(n^{12\alpha} \log N + n^{6\tau})$ & Exact\\\hline
% $\tt NetDag$ & any $Q$ & $O(n^{12\alpha} N\log N)$ & $O(n^{12\alpha} N \log N)$ &  $O(n^{12\alpha} \log N)$ & 2-opt\\\hline
% \end{tabular}
% \caption{Summary on related work ($N$ denotes the size of the database $\mathcal D$, $n$ denotes the size of the largest graph in $\mathcal D$, $\alpha$ and $\tau$ are constants, $R$ is the exact NN of $Q$ in $\mathcal D$, $^*$ means empirical analysis without rigorous proof)}
% \label{tab:sumRW}
% \end{table*}



%\noindent
\stab
{\bf Contributions.} The contributions of this paper are as follows.

\begin{itemize}[leftmargin=\parindent]
\item \lyu{We study the problem of approximate similarity search in graph databases.} \lv{We evaluate the GED space and observe that ......try to summarize the observation.} We observe that in the GED space, when the distance between graphs is relatively large, the space exhibits a doubling property with a small expansion rate. However, as the distance becomes smaller, the expansion rate grows dramatically. Leveraging this characteristic, we design corresponding indexing and search methods for the graph approximate similarity search.

\item We propose the hierarchical structure $\upperpart$, which utilizes the properties of a doubling space to accelerate \lv{how? elaborate a little} the coarse localization of queries.

\item We propose $\lowerpartfull$, which, based on the coarse localization by $\upperpart$, conducts a fine-grained search by exploiting the intrinsic properties of GED.
\item \lv{We conduct experimental evaluation on ...}
\end{itemize}


% When recall is 0.95 and 0.98, our method is about 2.7x to 47x and 2.6x to 25x faster than existing methods on well-known real-world datasets, respectively.

%\noindent
\stab
{\bf Organizations.} 
The rest of this paper is organized as follows. Section 2 discusses the related work. Section 3 introduces the preliminaries and problem statement. Section 4 gives an analysis of GED space. Section 5 and 6 introduce the two parts of our index. Updates of the index is presented in Section 7. Section 8 presents optimizations. Section 9 presents the experiments. Section 10 concludes the paper.

\begin{table*}[htbp]
\centering
\begin{threeparttable}
\caption{A Summary of some characteristics of index-based similarity search methods}
\begin{tabular}{|l|c|c|c|}
\hline
Methods & Query time complexity & Index size & Construction time complexity \\
\hline\hline
\textsc{NSW}
& $O(\log N)^*$ 
& $O(N)$ 
& $O(N\log N)^*$  \\
\hline
\textsc{HNSW}
& $O(\log N)^*$ 
& $O(N)$ 
& $O(N\log N)^*$  \\
\hline
\textsc{Cover Tree} 
& $O(c^{12}\log N)$ 
& $O(c^6 N)$ 
& $O(c^6 N\log N)$ \\
\hline
\textsc{Net-tree} 
& $O(\lambda^{O(1)}\log N)$ 
& $O(\lambda^{O(1)} N)$ 
& $O(\lambda^{O(1)} N\log N)$ \\
\hline
Layer $1$ of $\net$: $\upperpart$ 
& $C_G^6 \cdot \log_2 \phi$ 
& $N \cdot C_G^6 \cdot \log_2 \phi$ 
& $N \cdot C_G^6 \cdot \log_2 \phi$  \\
\hline
Layer $2$ of $\net$: $\lowerpartfull$ 
& $\sum_{i\leq \alpha}\binom{m}{i}\cdot C_A^{2\tau + 1}$ 
& $O(\alpha \cdot N)$ 
& $O(N)$  \\
\hline
\end{tabular}
\begin{tablenotes}
\item[*] means without guarantee.
\end{tablenotes}
\label{tab:complexity}
\end{threeparttable}
\end{table*}

